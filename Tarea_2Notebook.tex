% Options for packages loaded elsewhere
\PassOptionsToPackage{unicode}{hyperref}
\PassOptionsToPackage{hyphens}{url}
%
\documentclass[
]{article}
\usepackage{amsmath,amssymb}
\usepackage{iftex}
\ifPDFTeX
  \usepackage[T1]{fontenc}
  \usepackage[utf8]{inputenc}
  \usepackage{textcomp} % provide euro and other symbols
\else % if luatex or xetex
  \usepackage{unicode-math} % this also loads fontspec
  \defaultfontfeatures{Scale=MatchLowercase}
  \defaultfontfeatures[\rmfamily]{Ligatures=TeX,Scale=1}
\fi
\usepackage{lmodern}
\ifPDFTeX\else
  % xetex/luatex font selection
\fi
% Use upquote if available, for straight quotes in verbatim environments
\IfFileExists{upquote.sty}{\usepackage{upquote}}{}
\IfFileExists{microtype.sty}{% use microtype if available
  \usepackage[]{microtype}
  \UseMicrotypeSet[protrusion]{basicmath} % disable protrusion for tt fonts
}{}
\makeatletter
\@ifundefined{KOMAClassName}{% if non-KOMA class
  \IfFileExists{parskip.sty}{%
    \usepackage{parskip}
  }{% else
    \setlength{\parindent}{0pt}
    \setlength{\parskip}{6pt plus 2pt minus 1pt}}
}{% if KOMA class
  \KOMAoptions{parskip=half}}
\makeatother
\usepackage{xcolor}
\usepackage[margin=1in]{geometry}
\usepackage{color}
\usepackage{fancyvrb}
\newcommand{\VerbBar}{|}
\newcommand{\VERB}{\Verb[commandchars=\\\{\}]}
\DefineVerbatimEnvironment{Highlighting}{Verbatim}{commandchars=\\\{\}}
% Add ',fontsize=\small' for more characters per line
\usepackage{framed}
\definecolor{shadecolor}{RGB}{248,248,248}
\newenvironment{Shaded}{\begin{snugshade}}{\end{snugshade}}
\newcommand{\AlertTok}[1]{\textcolor[rgb]{0.94,0.16,0.16}{#1}}
\newcommand{\AnnotationTok}[1]{\textcolor[rgb]{0.56,0.35,0.01}{\textbf{\textit{#1}}}}
\newcommand{\AttributeTok}[1]{\textcolor[rgb]{0.13,0.29,0.53}{#1}}
\newcommand{\BaseNTok}[1]{\textcolor[rgb]{0.00,0.00,0.81}{#1}}
\newcommand{\BuiltInTok}[1]{#1}
\newcommand{\CharTok}[1]{\textcolor[rgb]{0.31,0.60,0.02}{#1}}
\newcommand{\CommentTok}[1]{\textcolor[rgb]{0.56,0.35,0.01}{\textit{#1}}}
\newcommand{\CommentVarTok}[1]{\textcolor[rgb]{0.56,0.35,0.01}{\textbf{\textit{#1}}}}
\newcommand{\ConstantTok}[1]{\textcolor[rgb]{0.56,0.35,0.01}{#1}}
\newcommand{\ControlFlowTok}[1]{\textcolor[rgb]{0.13,0.29,0.53}{\textbf{#1}}}
\newcommand{\DataTypeTok}[1]{\textcolor[rgb]{0.13,0.29,0.53}{#1}}
\newcommand{\DecValTok}[1]{\textcolor[rgb]{0.00,0.00,0.81}{#1}}
\newcommand{\DocumentationTok}[1]{\textcolor[rgb]{0.56,0.35,0.01}{\textbf{\textit{#1}}}}
\newcommand{\ErrorTok}[1]{\textcolor[rgb]{0.64,0.00,0.00}{\textbf{#1}}}
\newcommand{\ExtensionTok}[1]{#1}
\newcommand{\FloatTok}[1]{\textcolor[rgb]{0.00,0.00,0.81}{#1}}
\newcommand{\FunctionTok}[1]{\textcolor[rgb]{0.13,0.29,0.53}{\textbf{#1}}}
\newcommand{\ImportTok}[1]{#1}
\newcommand{\InformationTok}[1]{\textcolor[rgb]{0.56,0.35,0.01}{\textbf{\textit{#1}}}}
\newcommand{\KeywordTok}[1]{\textcolor[rgb]{0.13,0.29,0.53}{\textbf{#1}}}
\newcommand{\NormalTok}[1]{#1}
\newcommand{\OperatorTok}[1]{\textcolor[rgb]{0.81,0.36,0.00}{\textbf{#1}}}
\newcommand{\OtherTok}[1]{\textcolor[rgb]{0.56,0.35,0.01}{#1}}
\newcommand{\PreprocessorTok}[1]{\textcolor[rgb]{0.56,0.35,0.01}{\textit{#1}}}
\newcommand{\RegionMarkerTok}[1]{#1}
\newcommand{\SpecialCharTok}[1]{\textcolor[rgb]{0.81,0.36,0.00}{\textbf{#1}}}
\newcommand{\SpecialStringTok}[1]{\textcolor[rgb]{0.31,0.60,0.02}{#1}}
\newcommand{\StringTok}[1]{\textcolor[rgb]{0.31,0.60,0.02}{#1}}
\newcommand{\VariableTok}[1]{\textcolor[rgb]{0.00,0.00,0.00}{#1}}
\newcommand{\VerbatimStringTok}[1]{\textcolor[rgb]{0.31,0.60,0.02}{#1}}
\newcommand{\WarningTok}[1]{\textcolor[rgb]{0.56,0.35,0.01}{\textbf{\textit{#1}}}}
\usepackage{longtable,booktabs,array}
\usepackage{calc} % for calculating minipage widths
% Correct order of tables after \paragraph or \subparagraph
\usepackage{etoolbox}
\makeatletter
\patchcmd\longtable{\par}{\if@noskipsec\mbox{}\fi\par}{}{}
\makeatother
% Allow footnotes in longtable head/foot
\IfFileExists{footnotehyper.sty}{\usepackage{footnotehyper}}{\usepackage{footnote}}
\makesavenoteenv{longtable}
\usepackage{graphicx}
\makeatletter
\def\maxwidth{\ifdim\Gin@nat@width>\linewidth\linewidth\else\Gin@nat@width\fi}
\def\maxheight{\ifdim\Gin@nat@height>\textheight\textheight\else\Gin@nat@height\fi}
\makeatother
% Scale images if necessary, so that they will not overflow the page
% margins by default, and it is still possible to overwrite the defaults
% using explicit options in \includegraphics[width, height, ...]{}
\setkeys{Gin}{width=\maxwidth,height=\maxheight,keepaspectratio}
% Set default figure placement to htbp
\makeatletter
\def\fps@figure{htbp}
\makeatother
\setlength{\emergencystretch}{3em} % prevent overfull lines
\providecommand{\tightlist}{%
  \setlength{\itemsep}{0pt}\setlength{\parskip}{0pt}}
\setcounter{secnumdepth}{-\maxdimen} % remove section numbering
\ifLuaTeX
  \usepackage{selnolig}  % disable illegal ligatures
\fi
\IfFileExists{bookmark.sty}{\usepackage{bookmark}}{\usepackage{hyperref}}
\IfFileExists{xurl.sty}{\usepackage{xurl}}{} % add URL line breaks if available
\urlstyle{same}
\hypersetup{
  pdftitle={Tarea 2},
  hidelinks,
  pdfcreator={LaTeX via pandoc}}

\title{Tarea 2}
\author{}
\date{\vspace{-2.5em}}

\begin{document}
\maketitle

\hypertarget{anuxe1lisis-de-datos-i}{%
\subsubsection{Análisis de Datos I}\label{anuxe1lisis-de-datos-i}}

\emph{Enzo Loiza B. - 23 de Abril 2024}

\hypertarget{introducciuxf3n}{%
\subsection{Introducción}\label{introducciuxf3n}}

Consideramos un extracto de las respuestas de la Encuesta de
Caracterización Socioeconómica CASEN de 2017 adjunta en el archivo
\texttt{casen.dta} con las siguientes variables:

\begin{longtable}[]{@{}ll@{}}
\caption{Variables a utilizar de la Encuesta CASEN 2017}\tabularnewline
\toprule\noalign{}
Variables & Descripción \\
\midrule\noalign{}
\endfirsthead
\toprule\noalign{}
Variables & Descripción \\
\midrule\noalign{}
\endhead
\bottomrule\noalign{}
\endlastfoot
\texttt{region} & Número de la región \\
\texttt{comuna} & Código de la Comuna \\
\texttt{tot\_hog} & Total de hogares en la vivienda \\
\texttt{tot\_per} & Total de personas en el hogar \\
\texttt{tot\_nuc} & Total de núcleos en el hogar \\
\texttt{sexo} & 1 (Hombre), 2 (Mujer) \\
\texttt{edad} & Edad en años \\
\texttt{ecivil} & Estado civil \\
\texttt{e9te} & Tipo de enseñanza \\
\texttt{s5} & Edad al tener el primer hijo \\
\texttt{s12} & Sistema de salud al que pertenece \\
\texttt{qaut} & Quintil autónomo nacional \\
\end{longtable}

\hypertarget{pasos-previos}{%
\subsection{Pasos previos}\label{pasos-previos}}

Abrimos las librerías necesarias para el análisis.

\begin{Shaded}
\begin{Highlighting}[]
\FunctionTok{library}\NormalTok{(dplyr)}
\end{Highlighting}
\end{Shaded}

\begin{verbatim}
## 
## Attaching package: 'dplyr'
\end{verbatim}

\begin{verbatim}
## The following objects are masked from 'package:stats':
## 
##     filter, lag
\end{verbatim}

\begin{verbatim}
## The following objects are masked from 'package:base':
## 
##     intersect, setdiff, setequal, union
\end{verbatim}

\begin{Shaded}
\begin{Highlighting}[]
\FunctionTok{library}\NormalTok{(tidyverse)}
\end{Highlighting}
\end{Shaded}

\begin{verbatim}
## -- Attaching core tidyverse packages ------------------------ tidyverse 2.0.0 --
## v forcats   1.0.0     v readr     2.1.5
## v ggplot2   3.5.0     v stringr   1.5.1
## v lubridate 1.9.3     v tibble    3.2.1
## v purrr     1.0.2     v tidyr     1.3.1
\end{verbatim}

\begin{verbatim}
## -- Conflicts ------------------------------------------ tidyverse_conflicts() --
## x dplyr::filter() masks stats::filter()
## x dplyr::lag()    masks stats::lag()
## i Use the conflicted package (<http://conflicted.r-lib.org/>) to force all conflicts to become errors
\end{verbatim}

\begin{Shaded}
\begin{Highlighting}[]
\FunctionTok{library}\NormalTok{(ggplot2)}
\end{Highlighting}
\end{Shaded}

Luego leemos los datos de la encuesta y hacemos una primera vista de
ellos.

\begin{Shaded}
\begin{Highlighting}[]
\NormalTok{db }\OtherTok{\textless{}{-}} \FunctionTok{read.csv}\NormalTok{(}\StringTok{\textquotesingle{}casen.csv\textquotesingle{}}\NormalTok{)}
\end{Highlighting}
\end{Shaded}

\begin{Shaded}
\begin{Highlighting}[]
\FunctionTok{head}\NormalTok{(db)}
\end{Highlighting}
\end{Shaded}

\begin{verbatim}
##   region comuna tot_hog tot_per tot_nuc sexo edad ecivil e9te s5 s12 qaut
## 1      1   1101       1       1       1    2   56      8   NA NA  99    3
## 2      1   1101       1       1       1    2   21      8   NA NA   8    3
## 3      1   1101       1       2       1    1   24      2   NA NA  99    4
## 4      1   1101       1       2       1    1   28      2   NA NA  99    4
## 5      1   1101       1       3       1    1   26      1   NA 25   2    2
## 6      1   1101       1       3       1    2   26      1   NA 25   2    2
\end{verbatim}

\hypertarget{primer-anuxe1lisis}{%
\subsection{Primer análisis}\label{primer-anuxe1lisis}}

\textbf{a.} Filtramos usando el comando \texttt{filter} de la base de
datos, sólo las personas de la Región de Magallanes y la Antártica
Chilena. De acuerdo al Libro de Códigos de la CASEN, esta región
corresponde al número 12.

\begin{Shaded}
\begin{Highlighting}[]
\NormalTok{magallanes }\OtherTok{\textless{}{-}} \FunctionTok{filter}\NormalTok{(db, region }\SpecialCharTok{==} \DecValTok{12}\NormalTok{)}
\FunctionTok{head}\NormalTok{(magallanes)}
\end{Highlighting}
\end{Shaded}

\begin{verbatim}
##   region comuna tot_hog tot_per tot_nuc sexo edad ecivil e9te s5 s12 qaut
## 1     12  12101       1       2       1    1   62      1   NA 23   6    5
## 2     12  12101       1       2       1    2   61      1   NA 20   6    5
## 3     12  12101       1       3       1    1   64      1   NA 36   6    5
## 4     12  12101       1       3       1    2   28      8   NA NA   7    5
## 5     12  12101       1       3       1    2   55      1  998 27   6    5
## 6     12  12101       1       2       1    1   49      1   NA NA   6    5
\end{verbatim}

Luego, usamos la librería \texttt{dplyr} para, en cadena, hacer el
reporte de mínimo, máximo, media y desviación estándar de la variable
\texttt{tot\_hog} (número de hogares por vivienda) respecto de
\texttt{qaut} (quintil autónomo nacional).

\begin{Shaded}
\begin{Highlighting}[]
\NormalTok{resumen }\OtherTok{\textless{}{-}}\NormalTok{ magallanes }\SpecialCharTok{\%\textgreater{}\%}
  \FunctionTok{group\_by}\NormalTok{(qaut) }\SpecialCharTok{\%\textgreater{}\%}
  \FunctionTok{summarise}\NormalTok{(}
    \AttributeTok{Minimo =} \FunctionTok{min}\NormalTok{(tot\_hog),}
    \AttributeTok{Maximo =} \FunctionTok{max}\NormalTok{(tot\_hog),}
    \AttributeTok{Media =} \FunctionTok{mean}\NormalTok{(tot\_hog),}
    \AttributeTok{Desviacion\_Estandar =} \FunctionTok{sd}\NormalTok{(tot\_hog)}
\NormalTok{  )}

\NormalTok{resumen}
\end{Highlighting}
\end{Shaded}

\begin{verbatim}
## # A tibble: 6 x 5
##    qaut Minimo Maximo Media Desviacion_Estandar
##   <int>  <int>  <int> <dbl>               <dbl>
## 1     1      1      3  1.01               0.122
## 2     2      1      4  1.03               0.223
## 3     3      1      4  1.02               0.201
## 4     4      1      4  1.04               0.272
## 5     5      1      7  1.04               0.341
## 6    NA      1      1  1                  0
\end{verbatim}

\textbf{b.} Sin considerar los filtros anteriores, reportamos la edad
media de las personas al tener su primer hijo. Acá es importante restar
de la base de datos a quienes aparecen como
\texttt{NA\textquotesingle{}s}, pues corresponden en su mayoría a
personas que no (o aún no) tienen hijos.

Primero realizamos una vista de cómo se comportan los datos:

\begin{Shaded}
\begin{Highlighting}[]
\NormalTok{db}\SpecialCharTok{$}\NormalTok{sexo }\OtherTok{\textless{}{-}} \FunctionTok{as.factor}\NormalTok{(db}\SpecialCharTok{$}\NormalTok{sexo)}
\FunctionTok{ggplot}\NormalTok{(db, }\FunctionTok{aes}\NormalTok{(}\AttributeTok{x =}\NormalTok{ s5, }\AttributeTok{fill =}\NormalTok{ sexo)) }\SpecialCharTok{+}
  \FunctionTok{geom\_histogram}\NormalTok{()}
\end{Highlighting}
\end{Shaded}

\begin{verbatim}
## `stat_bin()` using `bins = 30`. Pick better value with `binwidth`.
\end{verbatim}

\begin{verbatim}
## Warning: Removed 97067 rows containing non-finite outside the scale range
## (`stat_bin()`).
\end{verbatim}

\includegraphics{Tarea_2Notebook_files/figure-latex/unnamed-chunk-6-1.pdf}

Esto nos indica que además de los 97067 casos sin edad reportada,
existen también algunos casos donde la edad se dispara. Generalmente los
99 también son instancias sin datos, por lo que es necesario verificar
si se trata de ellos.

\begin{Shaded}
\begin{Highlighting}[]
\NormalTok{db }\SpecialCharTok{\%\textgreater{}\%} \FunctionTok{filter}\NormalTok{(s5 }\SpecialCharTok{\textgreater{}} \DecValTok{75}\NormalTok{) }\SpecialCharTok{\%\textgreater{}\%} \CommentTok{\# filtramos por una edad prudente de 75}
  \FunctionTok{summarise}\NormalTok{(}
    \AttributeTok{min =} \FunctionTok{min}\NormalTok{(s5),}
    \AttributeTok{max =} \FunctionTok{max}\NormalTok{(s5),}
    \AttributeTok{mean =} \FunctionTok{mean}\NormalTok{(s5)}
\NormalTok{  )}
\end{Highlighting}
\end{Shaded}

\begin{verbatim}
##   min max mean
## 1  99  99   99
\end{verbatim}

Como se comprobó, que es así, ahora realizamos los filtros que
corresponden para obtener la media de edad:

\begin{Shaded}
\begin{Highlighting}[]
\NormalTok{primer\_hijo\_1 }\OtherTok{\textless{}{-}}\NormalTok{ db }\SpecialCharTok{\%\textgreater{}\%}
  \FunctionTok{filter}\NormalTok{(s5 }\SpecialCharTok{!=} \DecValTok{99}\NormalTok{) }\SpecialCharTok{\%\textgreater{}\%}
  \FunctionTok{summarise}\NormalTok{(}
    \AttributeTok{Media\_1 =} \FunctionTok{mean}\NormalTok{(s5, }\AttributeTok{na.rm =} \ConstantTok{TRUE}\NormalTok{)}
\NormalTok{)}

\NormalTok{primer\_hijo\_1}
\end{Highlighting}
\end{Shaded}

\begin{verbatim}
##    Media_1
## 1 23.59679
\end{verbatim}

Ahora filtramos seleccionando todas las mujeres casadas (de acuerdo al
Libro de Códigos, corresponde al 1) que tuvieron enseñanza básica. En el
mismo documento corresponde a las siguientes personas110).

\begin{Shaded}
\begin{Highlighting}[]
\NormalTok{primer\_hijo\_2 }\OtherTok{\textless{}{-}}\NormalTok{ db }\SpecialCharTok{\%\textgreater{}\%} \FunctionTok{filter}\NormalTok{(ecivil }\SpecialCharTok{==} \DecValTok{1}\NormalTok{) }\SpecialCharTok{\%\textgreater{}\%}
  \FunctionTok{filter}\NormalTok{(sexo }\SpecialCharTok{==} \DecValTok{2}\NormalTok{) }\SpecialCharTok{\%\textgreater{}\%}
  \FunctionTok{filter}\NormalTok{(e9te }\SpecialCharTok{!=} \DecValTok{10}\NormalTok{) }\SpecialCharTok{\%\textgreater{}\%}
  \FunctionTok{filter}\NormalTok{(e9te }\SpecialCharTok{!=} \DecValTok{77}\NormalTok{) }\SpecialCharTok{\%\textgreater{}\%}
  \FunctionTok{filter}\NormalTok{(s5 }\SpecialCharTok{!=} \DecValTok{99}\NormalTok{) }\SpecialCharTok{\%\textgreater{}\%}
  \FunctionTok{summarise}\NormalTok{(}
    \AttributeTok{Media\_2 =} \FunctionTok{mean}\NormalTok{(s5, }\AttributeTok{na.rm =} \ConstantTok{TRUE}\NormalTok{)}
\NormalTok{  )}

\NormalTok{primer\_hijo\_2}
\end{Highlighting}
\end{Shaded}

\begin{verbatim}
##    Media_2
## 1 22.11254
\end{verbatim}

El desarrollo de la

\hypertarget{pregunta-2}{%
\subsection{Pregunta 2}\label{pregunta-2}}

a.

\end{document}
